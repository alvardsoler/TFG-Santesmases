%---------------------------------------------------------------------
%
%                          Capítulo 10
%
%---------------------------------------------------------------------

\chapter{Aportaciones individuales}

\begin{FraseCelebre}
\begin{Frase}
Trabajar en equipo divide el trabajo y multiplica los resultados
\end{Frase}
\begin{Fuente}
Anónimo
\end{Fuente}
\end{FraseCelebre}


Lista de aportaciones individuales de cada uno de los miembros del equipo.

%-------------------------------------------------------------------
\section{Organización general del proyecto}
%-------------------------------------------------------------------
\label{cap10:sec:general}

En general nos hemos organizado de manera independiente. Cada uno de los miembros ha realizado uno de los minijuegos, aunque luego hemos desarrollado algunas funcionalidades juntos y otras, a parte del minijuego de cada uno, también de manera individual. Aun siendo cada uno de los minijuegos responsabilidad de uno de los miembros del equipo, nos hemos apoyado cuando teníamos problemas en el desarrollo individual de cada uno.

Para mejorar el trabajo en equipo hemos hecho uso del Sistema de control de versiones Git a través de la plataforma de GitHub \footnote{\href{https://github.com/rulo7/AUNGMENTED_REALITY_UNITY.git}{Repositorio utilizado}} . Se trabajaba en una rama de desarrollo sobre la que íbamos integrando aquellos cambios que generabamos sobre el proyecto y otra de producción, donde se registraban y etiquetaban, aquellas versiones finales del juego de cada fase del desarrollo.

\section{Raúl Cobos}
%-------------------------------------------------------------------
\label{cap10:sec:raul}

El videojuego que he desarrollado es el Arkanoid además de las siguientes aportaciones.

\begin{itemize}
\item{Realización de tutoriales en Unity3D para comprender en profundidad cómo funcionan sus escenas y sus mecánicas.}
\item{Autoaprendizaje e investigación de nuevas tecnologías para mi como era Vuforia.}
\item{Realización de prototipos de Unity3D y Vuforia.}
\item{Testeo con Álvar de diferentes maneras de interactuar con la RA en los primeros momentos.}
\item{Comunicación y reuniones con nuestro tutor Guillermo a través de reuniones presenciales y correos electrónicos.}
\item{Evaluación con usuarios e interpretación de éstas.}
\item{Escritura de todos los apartados de la memoria (menos los de los minijuegos de mis compañeros) y revisión de los contenidos del a misma.}
\item{Toma de decisiones con el resto de mis compañeros.}
\item{Construcción del entorno de persistencia de puntuaciones.}
\item{Adaptación del cuestionario SUS.}
\item{Integración de la extensión SALSA With RandomEyes para la animación del personaje de las escenas intermedias.}
\item{Adaptación responsiva de las fuentes de los textos del juego.}
\end{itemize}

\section{Álvar D. Soler}
%-------------------------------------------------------------------
\label{cap10:sec:alvar}

El minijuego que yo he desarrollado ha sido el Space Invader. Además de todo este minijuego, he llevado a cabo las siguientes tareas:

\begin{itemize}
\item{Realización de tutoriales en Unity3D para comprender en profundidad cómo funcionan sus escenas y sus mecánicas.}
\item{Autoaprendizaje e investigación de nuevas tecnologías para mi como era Vuforia.}
\item{Realización de prototipos de Unity3D y Vuforia.}
\item{Puesta en práctica de reconocimiento de texto propio en castellano.}
\item{Testeo con Raúl de diferentes maneras de interactuar con la RA en los primeros momentos.}
\item{Pruebas con Virtual Buttons para ver la viabilidad de utilizarlos en los minijuegos.}
\item{Animaciones de los Invaders.}
\item{Búsqueda de sonidos.}
\item{Realización de fotografías en el museo para poder orientarnos cuando trabajamos en nuestras casas.}
\item{Maquetación en LaTeX utilizando la plantilla de TeXiS.}
\item{Comunicación y reuniones con nuestro tutor Guillermo a través de reuniones presenciales y correos electrónicos.}
\item{Testeo para calibrar bien el tamaño de los Invaders y las Defensas con el cartel del exterior de la Facultad.}
\item{Evaluación con usuarios e interpretación de éstas.}
\item{Búsqueda de imágenes que fueran fáciles de reconocer por la cámara de Vuforia.}
\item{Escritura de todos los apartados de la memoria (menos los de los minijuegos de mis compañeros) y revisión de los contenidos del a misma.}
\item{Toma de decisiones con el resto de mis compañeros.}
\item{Traducción al inglés de los apartados de la memoria que están traducidos.}
\item{Testeo con smartphones Moto G 2013 y Orange Hi4G.}
\item{Mantenimiento del repositorio con la memoria en LaTeX.}
\end{itemize}

\section{María Picado} % (fold)
\label{cap10:sec:maria}

En mi caso, mi trabajo ha estado dividido en varias etapas, en cada una de las cuales me he dedicado a diferentes tareas.

\begin{itemize}

\item{La primera de ellas, fue junto a mis compañeros la realización de tutoriales en Unity3D para comprender y afianzar los conocimientos sobre el funcionamiento de esta herramienta.}

\item{Como ya dijimos anteriormente, aunque si conocíamos Unity, Vuforia era completamente desconocida para nosotros, con lo que mi siguiente tarea fue la de investigación y autoaprendizaje para conocer el funcionamiento de Vuforia. Y más adelante comencé con la creación de pequeños prototipos en Unity3d y Vuforia.}

\item{Durante todo el proceso de creación del proyecto, hemos mantenido los tres reuniones periódicas con nuestro director del TFG para ir mostrándole los avances y ponernos nuevos objetivos de cara a la siguiente reunión.}

\item{Una vez adquirimos los suficientes conocimientos para poder desenvolvernos tanto con Unity como con Vuforia, decidimos reunirnos los tres para diseñar el videojuego y decidir qué minijuegos implementaremos.}

\item{En esa reunión, una de las decisiones que tomamos fue la de que juegos íbamos a implementar cada uno, y a partir de ese momento me centré en la realización del minijuego Water Pipes. 
Lo primero que hice fue documentarme del juego original para ver cómo sería la mejor manera de adaptarlo a la RA.}

\item{Una de las cosas más importantes para que el juego se pudiera adaptar a la RA, era obtener la forma de que el usuario pudiera manipular las tuberías. Al ser una aplicación móvil, lo primero que intenté fue el ?DRAG AND DROP?, para que el jugador pulsara una tubería y la arrastrarse hasta donde quisiera cambiarla y al soltarla se cambiará automáticamente una tubería por la otra. Pero esta opción me dio bastantes problemas y entonces decidí que la forma en que se fueran colocando las tuberías fuera el intercambio entre ellas, pulsando sobre una e intercambiando por la siguiente en ser pulsada.}

\item{Una vez implementada la funcionalidad para colocar las tuberías, comencé a implementar la parte más importante del juego; el flujo del agua. Como existen 6 opciones de tuberías distintas y cada una de ellas tiene otras dos direcciones posibles por las que puede circular el agua. Tenía que buscar una forma de encapsular esa información para que lo único que nos preocupase fuera la salida de la tubería actual y la entrada de la siguiente. Por eso decidí usar el patrón de diseño Command.}

% imagen diagrama patrón command

\item{Al acabar la implementación del juego, el siguiente paso fue realizar el mayor número de test posibles a los usuarios. Los test los realizamos del juego completo, pero pedimos a los usuarios que puntuaron los juegos de manera independiente. Por eso, un vez que obtuvimos los primeros resultados, me dispuse a modificar el juego para implementar las mejoras que me aconsejaron los usuarios.}

\item{Durante todo el proyecto he colaborado en el desarrollo de esta memoria.}

\end{itemize}
% Variable local para emacs, para  que encuentre el fichero maestro de
% compilación y funcionen mejor algunas teclas rápidas de AucTeX
%%%
%%% Local Variables:
%%% mode: latex
%%% TeX-master: "../Tesis.tex"
%%% End:
