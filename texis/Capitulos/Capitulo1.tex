%---------------------------------------------------------------------
%
%                          Capítulo 1
%
%---------------------------------------------------------------------

\chapter{Introducción a \appname}

El proyecto que hemos desarrollado tiene como finalidad atraer al público al museo García Santesmases con una característica nueva y atractiva. Para esto, hemos utilizado la Realidad Aumentada (en adelante RA), y con ella, diseñado tres pequeños minijuegos que requieren poco tiempo para ser jugados y dan una visión nueva de lo que la RA y los videojuegos pueden aportar a un museo. Todo esto dentro de una aplicación para móviles Android.

\section{Museo de informática García-Santesmases}
%-------------------------------------------------------------------
\label{cap1:sec:museo}

Inaugurado en noviembre del 2003, el museo debe su nombre al físico, profesor y precursor de la informática española José García Santesmases, el cual fue catedrático de la Universidad Complutense. En él, se exponen máquinas desarrolladas por la UCM entre 1970 y 1950. Además, se exponen las computadoras comerciales del Centro de Cálculo de la UCM, aportaciones de particulares, los propios departamentos de la Universidad, etcétera \cite{migs}.

\figura{Tfg/santesmases.jpg}{width=.5\textwidth}{fig:museoGS}
	{Retrato de José García Santesmases por Eulogia Merle}

Además de computadoras, hay paneles explicativos que muestran información sobre éstas y sobre historia de la informática en general y cuenta también con gran cantidad de bibliografía presente en la biblioteca de la facultad.

El museo cuenta con dos plantas situadas en la 3ª y 4ª planta de la Facultad de Informática y su pieza más significativa es el ``Analizador diferencial electrónico'', diseñado por García Santesmases y es el primer computador desarrollado en España.

Las visitas al museo son libres y cualquiera puede acercarse a la Facultad de Informática y recorrer sus pasillos, pero si se prefiere ir en grupo, se puede concertar una cita a través del email del museo. Además, a lo largo del curso se organizan visitas guiadas por el museo.

Toda la información necesaria sobre el museo se puede encontrar en la web\footnote{\href{http://www.fdi.ucm.es/migs/}}. 

\section{Objetivos y motivación}
%-------------------------------------------------------------------
\label{cap1:sec:objetivos}

Nuestro objetivo es el desarrollo de una aplicación que mejorará la experiencia del usuario en un museo, en este caso, el museo de la Facultad de Informática ``García-Santesmases''. Pero no solo esto, sino que nuestro objetivo es hacerlo a través de los videojuegos y utilizando la RA.

Esto lo logramos mediante una ``yincana'', guiando al visitante a que recorra el museo en busca de misiones que tendrá que ir completando minijuegos para poder pasar a la siguiente misión. Así, al finalizar la visita, el jugador habrá recorrido el museo de una forma amena y divertida.

Nuestra motivación principal en la realización de este proyecto, fue profundizar en el desarrollo de videojuegos con una herramienta tan innovadora como lo es la RA. Para poder cumplir con los objetivos citados anteriormente, debemos antes centrarnos en investigar y recopilar información sobre distintos temas.
    
\begin{itemize}
\item{Para la realización de este proyecto, es fundamental conocer los trabajos ya existentes en museos con RA y más específicamente los que añaden a esto los videojuegos en su proyecto. Esta diferencia es importante, debido a que los museos están apostando por la RA de muchas formas diferentes. Pero al estar nuestro trabajo enfocado a los videojuegos, creemos importante centrarnos en este punto más detenidamente.}
\item{Existen diversas formas de incluir los videojuegos en un museo para dinamizar la visita al usuario. Como en nuestro proyecto decidimos hacer una yincana, otro objetivo que debemos cumplir es volver a investigar, pero esta vez, para recopilar la mayor información posible sobre cómo crear una yincana en un museo, cuál puede ser la mejor manera de ir guiando al visitante por el museo y cómo decidir qué partes del museo son las mejores para hacer captar la atención del visitante en ese lugar en concreto.}
\item{Otra decisión importante que tenemos que tomar es que minijuegos van a componer nuestra yincana. No podemos solo pensar en qué minijuegos queremos implementar, sino también debemos tener en cuenta los puntos anteriores para pensar en esto. Visitar el museo para ver en qué partes de éste se pueden incluir los juegos nos puede ayudar mucho para saber cuales son los juegos que queremos adaptar a la RA e incluir en nuestra yimcana.}
\end{itemize}

\section{Antecedentes}
%-------------------------------------------------------------------
\label{cap1:sec:antecedentes}

El proyecto que hemos realizado para este TFG es un proyecto nuevo y que ha sido diseñado e implementado desde el principio por nosotros. En años anteriores se realizaron trabajos de fin de grado dedicados a la RA en museos, como el realizado el año pasado (2014/2015) para el Museo de América \citep{RACMA}. Nuestro proyecto guarda muchas similitudes con el citado anteriormente, como el uso de RA en museos para mejorar la experiencia del visitante. Por tanto, este trabajo puede que sea nuestro antecedente, aunque el concepto de proyecto sea distinto, ya que ellos utilizaban la RA como medio de información, mientras que nosotros añadimos los videojuegos en RA como medio de entretenimiento en la visita.

% Variable local para emacs, para  que encuentre el fichero maestro de
% compilación y funcionen mejor algunas teclas rápidas de AucTeX
%%%
%%% Local Variables:
%%% mode: latex
%%% TeX-master: "../Tesis.tex"
%%% End:
