%---------------------------------------------------------------------
%
%                      resumen.tex
%
%---------------------------------------------------------------------
%
% Contiene el cap�tulo del resumen.
%
% Se crea como un cap�tulo sin numeraci�n.
%
%---------------------------------------------------------------------

\chapter{Resumen}
\cabeceraEspecial{Resumen}

La Realidad Aumentada es una tecnolog�a que combina im�genes reales con la superposici�n de im�genes virtuales. En esta memoria se detalla el trabajo hecho con esta tecnolog�a en la creaci�n de una yincana que, a trav�s de varios minijuegos busca dotar al Museo Garc�a Santesmases de un atractivo a�adido al de los objetos f�sicos ya expuestos. Veremos c�mo ha sido el proceso de desarrollo de la aplicaci�n, la toma de decisiones y los problemas que hemos encontrado. El nombre de la aplicaci�n es \appname, y est� disponible en Google Play Store para que cualquiera que visite el museo la pueda descargar y usar.

\textbf{Palabras clave}: Realidad Aumentada, Museos, Unity3D, Vuforia, Android, Videojuegos, yincana

\section*{Abstract} % mirar como hacer para poner bien el titulo de abstract

Augmented Reality is a technology that combines real images with overlapping virtual images. In this report we detail the work done with this technology in the creation of various mini-games to provide to the Museo Garc�a Santesmases an attractive addition to the exposed physical objects. We will see how was the application's development process, decision-making and the problems we have encountered. The application's name is \appname, and it is available in Play Store so anyone who visits the museum can download and use.

\textbf{Keywords}: Augmented Reality, Museums, Unity3D, Vuforia, Android, Videogames

\endinput
% Variable local para emacs, para  que encuentre el fichero maestro de
% compilaci�n y funcionen mejor algunas teclas r�pidas de AucTeX
%%%
%%% Local Variables:
%%% mode: latex
%%% TeX-master: "../Tesis.tex"
%%% End:
