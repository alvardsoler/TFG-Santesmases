%---------------------------------------------------------------------
%
%                      resumen.tex
%
%---------------------------------------------------------------------
%
% Contiene el cap�tulo del resumen.
%
% Se crea como un cap�tulo sin numeraci�n.
%
%---------------------------------------------------------------------

\chapter{Resumen}
\cabeceraEspecial{Resumen}

La Realidad Aumentada es una tecnolog�a que combina im�genes reales con la superposici�n de im�genes virtuales. En esta memoria se detalla el trabajo hecho con esta tecnolog�a en la creaci�n de varios minijuegos para dotar al Museo Garc�a Santesmases de un atractivo a�adido al de los objetos f�sicos ya expuestos. Veremos c�mo ha sido el proceso de desarrollo de la aplicaci�n, la toma de decisiones y los problemas que hemos encontrado. El nombre de la aplicaci�n es NOMBREAPP, y est� disponible en Play Store para que cualquiera que visite el museo la pueda descargar y usar.

Palabras clave: Realidad Aumentada, Museos, Unity3D, Vuforia, Android, Videojuegos

Abstract % mirar como hacer para poner bien el titulo de abstract

Lore ipsum en ingl�s.
Keywords: Aungmented Reality, Museums, Unity3D, Vuforia, Android, Videogames

\endinput
% Variable local para emacs, para  que encuentre el fichero maestro de
% compilaci�n y funcionen mejor algunas teclas r�pidas de AucTeX
%%%
%%% Local Variables:
%%% mode: latex
%%% TeX-master: "../Tesis.tex"
%%% End:
